% Template:     Template Tesis LaTeX
% Documento:    Configuración inicial del template
% Versión:      1.0.7 (01/09/2019)
% Codificación: UTF-8
%
% Autor: Pablo Pizarro R. @ppizarror
%        Facultad de Ciencias Físicas y Matemáticas
%        Universidad de Chile
%        pablo.pizarro@ing.uchile.cl, ppizarror.com
%
% Sitio web:    [https://latex.ppizarror.com/Template-Tesis/]
% Licencia MIT: [https://opensource.org/licenses/MIT]

\def\temaatratar{}
\def\titulodelinforme{\titulotesis}
\def\autordeldocumento{\autortesis}
\def\addemptypagetwosides{false}
\def\predocpageromannumber{true}
\def\predocresetpagenumber{true}
\def\indexnewpagec{false}
\def\indexnewpagef{false}
\def\indexnewpaget{false}
\def\showindex{true}
\def\showindexofcontents{true}
\def\codigodelcurso{}
\def\nombredelcurso{}
\checkvardefined{\autordeldocumento}
\checkvardefined{\departamentouniversidad}
\checkvardefined{\localizacionuniversidad}
\checkvardefined{\nombredelcurso}
\checkvardefined{\nombrefacultad}
\checkvardefined{\nombreuniversidad}
\checkvardefined{\temaatratar}
\checkvardefined{\titulodelinforme}
\makeatletter
	\g@addto@macro\autordeldocumento\xspace
	\g@addto@macro\codigodelcurso\xspace
	\g@addto@macro\departamentouniversidad\xspace
	\g@addto@macro\localizacionuniversidad\xspace
	\g@addto@macro\nombredelcurso\xspace
	\g@addto@macro\nombrefacultad\xspace
	\g@addto@macro\nombreuniversidad\xspace
	\g@addto@macro\temaatratar\xspace
	\g@addto@macro\titulodelinforme\xspace
\makeatother
\ifthenelse{\equal{\cfgpdfsecnumbookmarks}{true}}{
	\bookmarksetup{numbered}}{
}
\ifthenelse{\equal{\cfgshowbookmarkmenu}{true}}{
	\def\cdfpagemodepdf {UseOutlines}
	}{
	\def\cdfpagemodepdf {UseNone}
}
\hypersetup{
	bookmarksopen={\cfgpdfbookmarkopen},
	bookmarksopenlevel={\cfgbookmarksopenlevel},
	bookmarkstype={toc},
	pdfauthor={\autordeldocumento},
	pdfcenterwindow={\cfgpdfcenterwindow},
	pdfcopyright={\cfgpdfcopyright},
	pdfcreator={LaTeX},
	pdfdisplaydoctitle={\cfgpdfdisplaydoctitle},
	pdfencoding={unicode},
	pdffitwindow={\cfgpdffitwindow},
	pdfinfo={
		Curso.Codigo={\codigodelcurso},
		Curso.Nombre={\nombredelcurso},
		Documento.Autor={\autordeldocumento},
		Documento.Tema={\temaatratar},
		Documento.Titulo={\titulodelinforme},
		Template.Autor.Alias={ppizarror},
		Template.Autor.Email={pablo.pizarro@ing.uchile.cl},
		Template.Autor.Nombre={Pablo Pizarro R.},
		Template.Autor.Web={https://ppizarror.com/},
		Template.Codificacion={UTF-8},
		Template.Fecha={01/09/2019},
		Template.Latex.Compilador={pdflatex},
		Template.Licencia.Tipo={MIT},
		Template.Licencia.Web={https://opensource.org/licenses/MIT/},
		Template.Nombre={Template-Informe},
		Template.Tipo={Normal},
		Template.Version.Dev={1.0.7-N},
		Template.Version.Hash={E6BD26D1233C2E803893233A9144DF22},
		Template.Version.Release={1.0.7},
		Template.Web.Dev={https://github.com/Template-Latex/Template-Informe/},
		Template.Web.Manual={https://latex.ppizarror.com/Template-Informe/},
		Universidad.Departamento={\departamentouniversidad},
		Universidad.Nombre={\nombreuniversidad},
		Universidad.Ubicacion={\localizacionuniversidad}
	},
	pdfkeywords={\nombreuniversidad, \localizacionuniversidad},
	pdflang={\documentlang},
	pdfmenubar={\cfgpdfmenubar},
	pdfpagelayout={\cfgpdfpagemode},
	pdfpagemode={\cdfpagemodepdf},
	pdfproducer={Template-Informe v1.0.7 | (Pablo Pizarro R.) ppizarror.com},
	pdfremotestartview={Fit},
	pdfstartpage={1},
	pdfstartview={\cfgpdfpageview},
	pdfsubject={\temaatratar},
	pdftitle={\titulodelinforme},
	pdftoolbar={\cfgpdftoolbar}
}
\graphicspath{{./\defaultimagefolder}{./\defaultimagefolder/departamentos/}}
\renewcommand{\baselinestretch}{\defaultinterline}
\setlength{\headheight}{64 pt}
\setlength{\footnotemargin}{\marginfootnote pt}
\setlength{\columnsep}{\columnsepwidth em}
\ifthenelse{\equal{\showlinenumbers}{true}}{
	\setlength{\linenumbersep}{0.50cm}
	\renewcommand\linenumberfont{\normalfont\tiny\color{\linenumbercolor}}
	}{
}
\floatplacement{figure}{\imagedefaultplacement}
\floatplacement{table}{\tabledefaultplacement}
\floatplacement{tikz}{\tikzdefaultplacement}
\color{\maintextcolor}
\arrayrulecolor{\tablelinecolor}
\sethlcolor{\highlightcolor}
\ifthenelse{\equal{\showborderonlinks}{true}}{
	\hypersetup{
		citebordercolor=\numcitecolor,
		linkbordercolor=\linkcolor,
		urlbordercolor=\urlcolor
	}
}{
\hypersetup{
		hidelinks,
		colorlinks=true,
		citecolor=\numcitecolor,
		linkcolor=\linkcolor,
		urlcolor=\urlcolor
	}
}
\ifthenelse{\equal{\colorpage}{white}}{}{
	\pagecolor{\colorpage}
}
\setcaptionmargincm{\captionlrmargin}
\ifthenelse{\equal{\captiontextbold}{true}}{
	\renewcommand{\captiontextbold}{bf}}{
	\renewcommand{\captiontextbold}{}
}
\ifthenelse{\equal{\captiontextsubnumbold}{true}}{
	\renewcommand{\captiontextsubnumbold}{bf}}{
	\renewcommand{\captiontextsubnumbold}{}
}
\captionsetup{
	labelfont={color=\captioncolor, \captiontextbold},
	labelformat={\captionlabelformat},
	labelsep={\captionlabelsep},
	textfont={color=\captiontextcolor},
	singlelinecheck=on
}
\captionsetup*[subfigure]{
	labelfont={color=\captioncolor, \captiontextsubnumbold},
	labelformat={\subcaptionlabelformat},
	labelsep={\subcaptionlabelsep},
	textfont={color=\captiontextcolor},
	singlelinecheck=on
}
\captionsetup*[subtable]{
	labelfont={color=\captioncolor, \captiontextsubnumbold},
	labelformat={\subcaptionlabelformat},
	labelsep={\subcaptionlabelsep},
	textfont={color=\captiontextcolor},
	singlelinecheck=on
}
\floatsetup[figure]{
	captionskip=\captiontbmarginfigure pt
}
\floatsetup[table]{
	captionskip=\captiontbmargintable pt
}
\ifthenelse{\equal{\figurecaptiontop}{true}}{
	\floatsetup[figure]{position=above}}{
}
\ifthenelse{\equal{\tablecaptiontop}{true}}{
	\floatsetup[table]{position=top}
	}{
	\floatsetup[table]{position=bottom}
}
\ifthenelse{\equal{\captionalignment}{justified}}{
	\captionsetup{
		format=plain,
		justification=justified
	}
}{
\ifthenelse{\equal{\captionalignment}{centered}}{
	\captionsetup{
		justification=centering
	}
}{
\ifthenelse{\equal{\captionalignment}{left}}{
	\captionsetup{
		justification=raggedright,
		singlelinecheck=false
	}
}{
\ifthenelse{\equal{\captionalignment}{right}}{
	\captionsetup{
		justification=raggedleft,
		singlelinecheck=false
	}
}{
	\throwbadconfig{Posicion de leyendas desconocida}{\captionalignment}{justified,centered,left,right}}}}
}
\ifthenelse{\equal{\stylecitereferences}{natbib}}{
	\bibliographystyle{\natbibrefstyle}
	\setlength{\bibsep}{\natbibrefsep pt}
}{
\ifthenelse{\equal{\stylecitereferences}{apacite}}{
	\bibliographystyle{apacite}
	\setlength{\bibitemsep}{\apaciterefsep pt}
}{
\ifthenelse{\equal{\stylecitereferences}{bibtex}}{
	\bibliographystyle{apa}
	\newlength{\bibitemsep}
	\setlength{\bibitemsep}{.2\baselineskip plus .05\baselineskip minus .05\baselineskip}
	\newlength{\bibparskip}\setlength{\bibparskip}{0pt}
	\let\oldthebibliography\thebibliography
	\renewcommand\thebibliography[1]{
		\oldthebibliography{#1}
		\setlength{\parskip}{\bibitemsep}
		\setlength{\itemsep}{\bibparskip}
	}
	\setlength{\bibitemsep}{\bibtexrefsep pt}
}{
	\throwbadconfig{Estilo citas desconocido}{\stylecitereferences}{bibtex,apacite,natbib}}}
}
\patchcmd{\appendices}{\quad}{\sectionappendixlastchar\quad}{}{}
\begingroup
	\makeatletter
	\let\newcounter\@gobble\let\setcounter\@gobbletwo
	\globaldefs\@ne\let\c@loldepth\@ne
	\newlistof{listings}{lol}{\lstlistlistingname}
	\newlistentry{lstlisting}{lol}{0}
	\makeatother
\endgroup
\makeatletter
	\def\ifGm@preamble#1{\@firstofone}
	\appto\restoregeometry{
		\pdfpagewidth=\paperwidth
		\pdfpageheight=\paperheight}
	\apptocmd\newgeometry{
		\pdfpagewidth=\paperwidth
		\pdfpageheight=\paperheight}{}{}
\makeatother
\hfuzz=200pt
\vfuzz=200pt
\hbadness=\maxdimen
\vbadness=\maxdimen
\makeatletter
\preto\tabular{\global\rownum=\z@}
\preto\tabularx{\global\rownum=\z@}
\makeatother
\strictpagecheck
\titlespacing{\section}{0pt}{20pt}{10pt}
\titlespacing{\subsection}{0pt}{15pt}{10pt}
\ttfamily \hyphenchar\the\font=`\-
\urlstyle{\fonturl}
\pdfcompresslevel=\pdfcompilecompression
\pdfdecimaldigits=2
\pdfinclusionerrorlevel=0
\pdfminorversion=\pdfcompileversion
\pdfobjcompresslevel=\pdfcompileobjcompression
\setcounter{secnumdepth}{4}
\newcounter{subsubsubsection}[subsubsection]
\ifthenelse{\equal{\showdotaftersnum}{true}}{
	\renewcommand{\thesubsubsubsection}{\thesubsubsection.\arabic{subsubsubsection}.}
	\renewcommand{\theparagraph}{\thesubsubsubsection.\arabic{paragraph}.}
}{
	\renewcommand{\thesubsubsubsection}{\thesubsubsection.\arabic{subsubsubsection}}
	\renewcommand{\theparagraph}{\thesubsubsubsection.\arabic{paragraph}}
}
\makeatletter
	\def\toclevel@subsubsubsection{4}
	\def\toclevel@paragraph{5}
	\def\toclevel@subparagraph{6}
	\renewcommand\paragraph{\@startsection{paragraph}{5}{\z@}
		{3.25ex \@plus 1ex \@minus .2ex}
		{-1em}
		{\normalfont\normalsize\bfseries}}
	\renewcommand\subparagraph{\@startsection{subparagraph}{6}{\parindent}
		{3.25ex \@plus 1ex \@minus .2ex}
		{-1em}
		{\normalfont\normalsize\bfseries}}
	\ifthenelse{\equal{\showdotaftersnum}{true}}{
\def\l@subsubsubsection{\@dottedtocline{4}{7.83em}{4.15em}}
		\def\l@paragraph{\@dottedtocline{5}{11.98em}{4.92em}}
		\def\l@subparagraph{\@dottedtocline{6}{14.65em}{5.69em}}
	}{
		\def\l@subsubsubsection{\@dottedtocline{4}{6.97em}{4em}}
		\def\l@paragraph{\@dottedtocline{5}{10.97em}{5em}}
		\def\l@subparagraph{\@dottedtocline{6}{14em}{6em}}
	}
\makeatother
\setcounter{tocdepth}{\indexdepth}
\ifthenelse{\equal{\footnoterestart}{none}}{
}{
\ifthenelse{\equal{\footnoterestart}{sec}}{
	\counterwithin*{footnote}{section}
}{
\ifthenelse{\equal{\footnoterestart}{ssec}}{
	\counterwithin*{footnote}{subsection}
}{
\ifthenelse{\equal{\footnoterestart}{sssec}}{
	\counterwithin*{footnote}{subsubsection}
}{
\ifthenelse{\equal{\footnoterestart}{ssssec}}{
	\counterwithin*{footnote}{subsubsubsection}
}{
\ifthenelse{\equal{\footnoterestart}{page}}{
	\counterwithin*{footnote}{page}
}{
\ifthenelse{\equal{\footnoterestart}{chap}}{
	\counterwithin*{footnote}{chapter}
}{
	\throwbadconfig{Formato reinicio numero footnote desconocido}{\footnoterestart}{none,chap,page,sec,ssec,sssec,ssssec}}}}}}}
}
\ifthenelse{\equal{\equationrestart}{none}}{
}{
\ifthenelse{\equal{\equationrestart}{chap}}{
}{
\ifthenelse{\equal{\equationrestart}{sec}}{
}{
\ifthenelse{\equal{\equationrestart}{ssec}}{
}{
\ifthenelse{\equal{\equationrestart}{sssec}}{
}{
\ifthenelse{\equal{\equationrestart}{ssssec}}{
}{
	\throwbadconfig{Formato reinicio numero ecuacion desconocido}{\equationrestart}{none,chap,sec,ssec,sssec,ssssec}}}}}}
}
\newtheoremstyle{miestilo}{\baselineskip}{3pt}{\itshape}{}{\bfseries}{}{.5em}{}
\newtheoremstyle{miobs}{\baselineskip}{3pt}{}{}{\bfseries}{}{.5em}{}
\theoremstyle{miestilo}
\ifthenelse{\equal{\showsectioncaptionmat}{none}}{
	\newtheorem{defn}{\namemathdefn}
	\newtheorem{teo}{\namemaththeorem}
	\newtheorem{cor}{\namemathcol}
	\newtheorem{lema}{\namemathlem}
	\newtheorem{prop}{\namemathprp}
}{
\ifthenelse{\equal{\showsectioncaptionmat}{chap}}{
	\newtheorem{defn}{\namemathdefn}[chapter]
	\newtheorem{teo}{\namemaththeorem}[chapter]
	\newtheorem{cor}{\namemathcol}[chapter]
	\newtheorem{lema}{\namemathlem}[chapter]
	\newtheorem{prop}{\namemathprp}[chapter]
}{
\ifthenelse{\equal{\showsectioncaptionmat}{sec}}{
	\newtheorem{defn}{\namemathdefn}[section]
	\newtheorem{teo}{\namemaththeorem}[section]
	\newtheorem{cor}{\namemathcol}[section]
	\newtheorem{lema}{\namemathlem}[section]
	\newtheorem{prop}{\namemathprp}[section]
}{
\ifthenelse{\equal{\showsectioncaptionmat}{ssec}}{
	\newtheorem{defn}{\namemathdefn}[subsection]
	\newtheorem{teo}{\namemaththeorem}[subsection]
	\newtheorem{cor}{\namemathcol}[subsection]
	\newtheorem{lema}{\namemathlem}[subsection]
	\newtheorem{prop}{\namemathprp}[subsection]
}{
\ifthenelse{\equal{\showsectioncaptionmat}{sssec}}{
	\newtheorem{defn}{\namemathdefn}[subsubsection]
	\newtheorem{teo}{\namemaththeorem}[subsubsection]
	\newtheorem{cor}{\namemathcol}[subsubsection]
	\newtheorem{lema}{\namemathlem}[subsubsection]
	\newtheorem{prop}{\namemathprp}[subsubsection]
}{
\ifthenelse{\equal{\showsectioncaptionmat}{ssssec}}{
	\newtheorem{defn}{\namemathdefn}[subsubsubsection]
	\newtheorem{teo}{\namemaththeorem}[subsubsubsection]
	\newtheorem{cor}{\namemathcol}[subsubsubsection]
	\newtheorem{lema}{\namemathlem}[subsubsubsection]
	\newtheorem{prop}{\namemathprp}[subsubsubsection]
}{
	\throwbadconfig{Valor configuracion incorrecto}{\showsectioncaptionmat}{none,chap,sec,ssec,sssec,ssssec}}}}}}
}
\theoremstyle{miobs}
\newtheorem*{ej}{\namemathej}
\newtheorem*{obs}{\namemathobs}
\unaccentedoperators
\AtEndDocument{
	\addtocounter{equation}{\value{templateEquations}}
	\addtocounter{figure}{\value{templateFigures}}
	\addtocounter{lstlisting}{\value{templateListings}}
	\addtocounter{table}{\value{templateTables}}
}
