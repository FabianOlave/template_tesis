% Template:     Template Tesis LaTeX
% Documento:    Configuraciones del template
% Versión:      1.1.5 (17/09/2019)
% Codificación: UTF-8
%
% Autor: Pablo Pizarro R. @ppizarror
%        Facultad de Ciencias Físicas y Matemáticas
%        Universidad de Chile
%        pablo.pizarro@ing.uchile.cl, ppizarror.com
%
% Sitio web:    [https://latex.ppizarror.com/Template-Tesis/]
% Licencia MIT: [https://opensource.org/licenses/MIT]

% CONFIGURACIONES GENERALES
\def\addemptypagespredoc {false}   % Añade pag. en blanco después de portada,etc.
\def\defaultinterline {1.0}        % Interlineado por defecto [pt]
\def\defaultnewlinesize {12}       % Tamaño del salto de línea [pt]
\def\documentlang {es-CL}          % Define el idioma del documento
\def\fontdocument {lmodern}        % Tipografía base, ver soportadas en manual
\def\fonttypewriter {tmodern}      % Tipografía de \texttt, ver manual online
\def\fonturl {tt}                  % Tipo de fuente url {tt,sf,rm,same}
\def\importtikz {false}            % Utilizar la librería tikz
\def\pointdecimal {true}           % N° decimales con punto en vez de coma
\def\predocpageromanupper {false}  % Páginas en número romano en mayúsculas
\def\showlinenumbers {false}       % Muestra los números de línea del documento

% ESTILO HEADER-FOOTER
\def\disablehfrightmark {false}    % Desactiva el rightmark del header-footer
\def\hfstyle {style7}              % Estilo header-footer (16 estilos)
\def\hfwidthcourse {0.35}          % Tamaño máximo del curso en header-footer
\def\hfwidthtitle {0.6}            % Tamaño máximo del título en header-footer
\def\hfwidthwrap {false}           % Activa el tamaño máximo en header-footer

% CONFIGURACIÓN DE LAS LEYENDAS - CAPTION
\def\captionalignment {justified}  % Posición {centered,justified,left,right}
\def\captionlabelformat {simple}   % Formato leyenda {empty,simple,parens}
\def\captionlabelsep {colon}       % Sep. {none,colon,period,space,quad,newline}
\def\captionlessmarginimage {0.1}  % Margen sup/inf de fig. si no hay ley. [cm]
\def\captionlrmargin {2.0}         % Márgenes izq/der de la leyenda [cm]
\def\captionmarginmultimg {0.0}    % Margen izq/der leyendas múltiple img [cm]
\def\captionnumcode {arabic}       % N° código {arabic,alph,Alph,roman,Roman}
\def\captionnumequation {arabic}   % N° ecuaciones {arabic,alph,Alph,roman,Roman}
\def\captionnumfigure {arabic}     % N° figuras {arabic,alph,Alph,roman,Roman}
\def\captionnumsubfigure {alph}    % N° subfiguras {arabic,alph,Alph,roman,Roman}
\def\captionnumsubtable {alph}     % N° subtabla {arabic,alph,Alph,roman,Roman}
\def\captionnumtable {arabic}      % N° tabla {arabic,alph,Alph,roman,Roman}
\def\captiontbmarginfigure {9.35}  % Margen sup/inf de la leyenda en fig. [pt]
\def\captiontbmargintable {7.0}    % Margen sup/inf de la leyenda en tab. [pt]
\def\captiontextbold {false}       % Etiqueta (código,figura,tabla) en negrita
\def\captiontextsubnumbold {false} % N° subfigura/subtabla en negrita
\def\codecaptiontop {true}         % Leyenda arriba del código fuente
\def\figurecaptiontop {false}      % Leyenda arriba de las imágenes
\def\sectioncaptiondelimiter {.}   % Carácter delimitador n° objeto y sección
\def\showsectioncaptioncode {chap} % N° sec. código {none,chap,(s/ss/sss/ssss)ec}
\def\showsectioncaptioneqn {chap}  % N° sec. ecuación {none,chap,(s/ss/sss/ssss)ec}
\def\showsectioncaptionfig {chap}  % N° sec. figuras {none,chap,(s/ss/sss/ssss)ec}
\def\showsectioncaptionmat {chap}  % N° matemático {none,chap,(s/ss/sss/ssss)ec}
\def\showsectioncaptiontab {chap}  % N° sec. tablas {none,chap,(s/ss/sss/ssss)ec}
\def\subcaptionlabelformat{parens} % Formato leyenda sub. {empty,simple,parens}
\def\subcaptionlabelsep {space}    % Sep. {none,colon,period,space,quad,newline}
\def\tablecaptiontop {true}        % Leyenda arriba de las tablas

% CONFIGURACIÓN DEL ÍNDICE
\def\addindextobookmarks {true}    % Añade el índice a los marcadores del pdf
\def\charafterobjectindex {.}      % Carácter después de n° figura,tabla,código
\def\charnumpageindex {.}          % Carácter número de página en índice
\def\indexdepth {4}                % Profundidad máxima del índice
\def\indexstyle {tf}               % Estilo índice {f:figura, t:tabla, c:código}
\def\indextitlemargin {11.4}       % Margen título índice \insertindextitle [pt]
\def\objectchaptermargin {false}   % Activa margen de objetos entre capítulos
\def\objectindexindent {false}     % Indenta la lista de objetos
\def\showappendixsecindex {false}  % Título de la sec. de anexos en el índice

% ANEXO, CITAS, REFERENCIAS
\def\apaciterefsep {9}             % Separación entre refs. {apacite} [pt]
\def\appendixindepobjnum {true}    % Anexo usa n° objetos independientes
\def\bibtexrefsep {6}              % Separación entre refs. {bibtex} [pt]
\def\natbibnumbers {true}          % Forza el uso de números en la bibliografía
\def\natbibrefsep {6}              % Separación entre referencia {natbib} [pt]
\def\natbibrefstyle {plainnat}     % Formato de ref. natbib {apa,ieeetr,etc...}
\def\natbibsquare {true}           % Usa [] o () en las numeraciones
\def\sectionappendixlastchar {.}   % Carácter entre n° de sec. anexo y título
\def\addabstracttobookmarks {true} % Añade el resumen a los marcadores del pdf
\def\addagradectobookmarks {true}  % Añade el agradecimiento a los marcadores
\def\stylecitereferences {natbib}  % Estilo cita/ref. {apacite,bibtex,natbib}

% CONFIGURACIONES DE OBJETOS
\def\columnhspace {-0.4}           % Margen horizontal entre obj. \createcolumn
\def\columnsepwidth {2.1}          % Separación entre columnas [em]
\def\defaultimagefolder {img/}     % Carpeta raíz de las imágenes
\def\equationleftalign {false}     % Ecuaciones alineadas a la izquierda
\def\equationrestart {none}        % Reinicio n° {none,chap,(s/ss/sss/ssss)ec}
\def\footnotepagetoprule {false}   % Footnote en pag. tienen separador superior
\def\footnoterestart {none}        % N° footnote {none,chap,page,(s/ss/sss/ssss)ec}
\def\imagedefaultplacement {H}     % Posición por defecto de las imágenes
\def\marginalignbottom {-0.30}     % Margen inferior entorno align [cm]
\def\marginaligncaptbottom {0.05}  % Margen inferior entorno align caption[cm]
\def\marginaligncapttop {-0.60}    % Margen superior entorno align caption [cm]
\def\marginalignedbottom {-0.30}   % Margen inferior entorno aligned [cm]
\def\marginalignedcaptbottom {0.0} % Margen inferior entorno aligned caption[cm]
\def\marginalignedcapttop {-0.60}  % Margen superior entorno aligned caption[cm]
\def\marginalignedtop {-0.40}      % Margen superior entorno aligned [cm]
\def\marginaligntop {-0.40}        % Margen superior entorno align [cm]
\def\margineqncaptionbottom {0.0}  % Margen inferior caption ecuación [cm]
\def\margineqncaptiontop {-0.65}   % Margen superior caption ecuación [cm]
\def\marginequationbottom {-0.15}  % Margen inferior ecuaciones [cm]
\def\marginequationtop {0.0}       % Margen superior ecuaciones [cm]
\def\marginfloatimages {-13.0}     % Margen sup. fig. insertimageleft/right [pt]
\def\marginfootnote {10.0}         % Margen derecho footnote [pt]
\def\margingatherbottom {-0.20}    % Margen inferior entorno gather [cm]
\def\margingathercaptbottom {0.05} % Margen inferior entorno gather caption [cm]
\def\margingathercapttop {-0.77}   % Margen superior entorno gather [cm]
\def\margingatheredbottom {-0.10}  % Margen inf. entorno gathered [cm]
\def\margingatheredcaptbottom{0.0} % Margen inf. entorno gathered caption [cm]
\def\margingatheredcapttop {-0.77} % Margen superior entorno gathered [cm]
\def\margingatheredtop {-0.40}     % Margen superior entorno gathered [cm]
\def\margingathertop {-0.40}       % Margen superior entorno gather [cm]
\def\marginimagebottom {-0.15}     % Margen inferior figura [cm]
\def\marginimagemultright {0.50}   % Margen derecho imágenes múltiples [cm]
\def\marginimagemulttop {-0.30}    % Margen superior imágenes múltiples [cm]
\def\marginimagetop {0.0}          % Margen superior figuras [cm]
\def\numberedequation {true}       % Ecuaciones con \insert... numeradas
\def\sourcecodefontf {\ttfamily}   % Tipo de letra código fuente
\def\sourcecodefonts {\small}      % Tamaño letra código fuente
\def\sourcecodenumbersep {6}       % Separación entre número línea y código [pt]
\def\sourcecodetabsize {3}         % Tamaño tabulación código fuente
\def\tabledefaultplacement {H}     % Posición por defecto de las tablas
\def\tablepaddingh {0.85}          % Espaciado horizontal de celda de las tablas
\def\tablepaddingv {1.05}          % Espaciado vertical de celda de las tablas
\def\tikzdefaultplacement {H}      % Posición por defecto de las figuras tikz

% CONFIGURACIÓN DE LOS TÍTULOS
\def\anumsecaddtocounter {false}   % Insertar títulos anum. aumenta n° de sec
\def\fontsizessstitle{\normalsize} % Tamaño sub-sub-subtítulos
\def\fontsizesubsubtitle {\normalsize} % Tamaño sub-subtítulos
\def\fontsizesubtitle {\large}     % Tamaño subtítulos
\def\fontsizetitle {\Large}        % Tamaño títulos
\def\showdotaftersnum {true}       % Punto al final de n° (s/ss/sss/ssss)ection
\def\stylessstitle {\bfseries}     % Estilo sub-sub-subtítulos
\def\stylesubsubtitle {\bfseries}  % Estilo sub-subtítulos
\def\stylesubtitle {\bfseries}     % Estilo subtítulos
\def\styletitle {\bfseries}        % Estilo títulos

% CONFIGURACIÓN DE LOS COLORES DEL DOCUMENTO
\def\captioncolor {black}          % Color de la leyenda (código,figura,tabla)
\def\captiontextcolor {black}      % Color de la leyenda
\def\colorpage {white}             % Color de la página
\def\highlightcolor {yellow}       % Color del subrayado con \hl
\def\indextitlecolor {black}       % Color de los títulos del índice
\def\linenumbercolor {gray}        % Color del n° de línea (\showlinenumbers)
\def\linkcolor {black}             % Color de los links del documento
\def\maintextcolor {black}         % Color principal del texto
\def\numcitecolor {black}          % Color del n° de las referencias o citas
\def\portraittitlecolor {black}    % Color de los títulos de la portada
\def\showborderonlinks {false}     % Color de un link por un recuadro de color
\def\sourcecodebgcolor {lgray}     % Color de fondo del código fuente
\def\ssstitlecolor {black}         % Color de los sub-sub-subtítulos
\def\subsubtitlecolor {black}      % Color de los sub-subtítulos
\def\subtitlecolor {black}         % Color de los subtítulos
\def\tablelinecolor {black}        % Color de las líneas de las tablas
\def\tablerowfirstcolor {none}     % Primer color de celda de las tablas
\def\tablerowsecondcolor {gray!20} % Segundo color de celda de las tablas
\def\titlecolor {black}            % Color de los títulos
\def\urlcolor {magenta}            % Color de los enlaces web (\href,\url)

% MÁRGENES DE PÁGINA
\def\pagemarginbottom {2.0}        % Margen inferior página [cm]
\def\pagemarginleft {3.0}          % Margen izquierdo página [cm]
\def\pagemarginleftportrait {2.5}  % Margen izquierdo página portada [cm]
\def\pagemarginright {2.0}         % Margen derecho página [cm]
\def\pagemargintop {2.0}           % Margen superior página [cm]

% OPCIONES DEL PDF COMPILADO
\def\cfgbookmarksopenlevel {0}     % Nivel marcadores en pdf (1:secciones)
\def\cfgpdfbookmarkopen {true}     % Expande marcadores del nivel configurado
\def\cfgpdfcenterwindow {true}     % Centra ventana del lector al abrir el pdf
\def\cfgpdfcopyright {}            % Establece el copyright del documento
\def\cfgpdfdisplaydoctitle {true}  % Muestra título del informe en visor
\def\cfgpdffitwindow {false}       % Ajusta la ventana del lector tamaño pdf
\def\cfgpdfmenubar {true}          % Muestra el menú del lector
\def\cfgpdfpagemode {OneColumn}    % Modo de página {OneColumn,SinglePage}
\def\cfgpdfpageview {FitH}         % {Fit,FitH,FitV,FitR,FitB,FitBH,FitBV}
\def\cfgpdfsecnumbookmarks {false} % Número de la sec. en marcadores del pdf
\def\cfgpdftoolbar {true}          % Muestra barra de herramientas lector pdf
\def\cfgshowbookmarkmenu {true}    % Muestra menú marcadores al abrir el pdf
\def\pdfcompilecompression {9}     % Factor de compresión del pdf (0-9)
\def\pdfcompileobjcompression {2}  % Nivel compresión objetos del pdf (0-3)
\def\pdfcompileversion {7}         % Versión mínima del pdf compilado

% NOMBRE DE OBJETOS
\def\nameabstract {Resumen}           % Nombre del resumen-abstract
\def\nameagradec {Agradecimientos}    % Nombre del cap. de agradecimientos
\def\nameappendixsection {Anexos}     % Nombre de los anexos
\def\namemathcol {Corolario}          % Nombre de los colorarios
\def\namemathdefn {Definición}        % Nombre de las definiciones
\def\namemathej {Ejemplo}             % Nombre de los ejemplos
\def\namemathlem {Lema}               % Nombre de los lemas
\def\namemathobs {Observación}        % Nombre de las observaciones
\def\namemathprp {Proposición}        % Nombre de las proposiciones
\def\namemaththeorem {Teorema}        % Nombre de los teoremas
\def\nameportraitpage {Portada}       % Etiqueta página de la portada
\def\namereferences {Bibliografía}    % Nombre de la sección de referencias
\def\nomchapter {Capítulo}            % Nombre de los capítulos
\def\nomltappendixsection {Anexo}     % Etiqueta sección en anexo/apéndices
\def\nomltcont {Tabla de Contenidos}  % Nombre del índice de contenidos
\def\nomltfigure {Índice de Ilustraciones} % Nombre de la lista de figuras
\def\nomltsrc {Índice de Códigos}     % Nombre de la lista de código
\def\nomlttable {Índice de Tablas}    % Nombre de la lista de tablas
\def\nomltwfigure {Figura}            % Etiqueta leyenda de las figuras
\def\nomltwsrc {Código}               % Etiqueta leyenda del código fuente
\def\nomltwtable {Tabla}              % Etiqueta leyenda de las tablas
\def\nomnpageof { de }                % Etiqueta página # de #
